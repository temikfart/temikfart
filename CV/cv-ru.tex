\documentclass[10pt,a4paper,ragged2e,withhyper]{custom-altacv}

% Specify the CV version according to format: v.YY.MM.DD-L.S
% where <L> is language (RU, EN, etc.) and <S> is a special mark like 'demo'.
\newcommand{\cvversion}{v.23.9.6-RU-mixed}
% Change the page layout if you need to
\geometry{left=1cm,right=1cm,top=1.5cm,bottom=1cm,columnsep=1.2cm}

% The paracol package lets you typeset columns of text in parallel
\usepackage{paracol}

% Change the font if you want to, depending on whether
% you're using pdflatex or xelatex/lualatex
\ifxetexorluatex
% If using xelatex or lualatex:
\setmainfont{Roboto Slab}
\setsansfont{Lato}
\renewcommand{\familydefault}{\sfdefault}
\else
% If using pdflatex:
\usepackage[rm]{roboto}
\usepackage[defaultsans]{lato}
% \usepackage{sourcesanspro}
\renewcommand{\familydefault}{\sfdefault}
\fi

% Change the colours if you want to
\definecolor{SlateGrey}{HTML}{2E2E2E}
\definecolor{LightGrey}{HTML}{666666}
\definecolor{DarkBlue}{HTML}{2F69A2}
\definecolor{Blue}{HTML}{4785C3}
\definecolor{GoldenEarth}{HTML}{E7D192}
\colorlet{name}{black}
\colorlet{tagline}{Blue}
\colorlet{heading}{DarkBlue}
\colorlet{headingrule}{GoldenEarth}
\colorlet{subheading}{Blue}
\colorlet{accent}{Blue}
\colorlet{emphasis}{SlateGrey}
\colorlet{body}{LightGrey}

% Change some fonts, if necessary
\renewcommand{\namefont}{\Huge\rmfamily\bfseries}
\renewcommand{\personalinfofont}{\footnotesize}
\renewcommand{\cvsectionfont}{\LARGE\rmfamily\bfseries}
\renewcommand{\cvsubsectionfont}{\large\bfseries}


% Change the bullets for itemize and rating marker
% for \cvskill if you want to
\renewcommand{\itemmarker}{{\small\textbullet}}
\renewcommand{\ratingmarker}{\faCircle}

% Version watermark in the corner
\usepackage[angle=0, fontsize=0.015\paperwidth, text=\cvversion, hpos=0.99\paperwidth, vpos=0.01\paperwidth, anchor=rt]{draftwatermark}

% Text inside mathmode
\usepackage{amsmath}

% Huawei logo
\newcommand*\huawei{\raisebox{-0.15\totalheight}{\includegraphics[width=15pt]{images/huawei}}\hspace{.5ex}}



\begin{document}

\name{Фартыгин Артём}
\tagline{Software Engineer}

\personalinfo{%
  \email{fartygin.a@phystech.edu}
  \phone{+7 963 061 20 44}
  \telegram{temikfart}
  \github{temikfart}
%  \printinfo{\faLinkedin}{Фартыгин Артём}[https://linkedin.com/in/artyom-fartygin-61890024a]
  \location{Россия, Долгопрудный}
}

\makecvheader

%% Depending on your tastes, you may want to make fonts of itemize environments slightly smaller
% \AtBeginEnvironment{itemize}{\small}

%% Set the left/right column width ratio to 6:4.
\columnratio{0.6}

% Start a 2-column paracol. Both the left and right columns will automatically break across pages
% if things get too long.
\begin{paracol}{2}

\cvsection{Опыт Работы}

\cvjob{Лаборант \cvtag{Java}}{Институт Системного Программирования РАН -- Отдел компиляторных технологий}{Октябрь 2022 -- по н.в.}{Svace}{https://www.ispras.ru/technologies/svace/}
\begin{itemize}
	\item Разрабатываю систему защиты дистрибутива нашего продукта с использованием физических ключей;
	\item Разработал \textbf{6} детекторов небезопасного кода для основого статического анализатора компании Samsung --- \textbf{Svace}. Наработки включены в release-3.3.2.
\end{itemize}

\divider

\cvjob{Ассистент Инженер \cvtag{C++}}{\huawei Huawei -- System Engineering Lab}{Летняя стажировка 2022}{Huawei RRI}{https://career.huawei.ru/rri/ru/}

\begin{itemize}
	\item Разработал \textbf{3} детектора неэффективного кода (\textbf{Static Analysis}). Два из них анализируют \textbf{LLVM IR} для поиска идиом (\textbf{Idiom Detection}). Все программы используются в компании.
\end{itemize}

\divider

\cvjob{Backend Разработчик \cvtag{Ruby on Rails}}{ООО <<ТелеПроводник>>}{3 месяца}{ТелеПроводник}{https://teleprovodnik.ru/}

\begin{itemize}
	\item Модернизировал геопортал, который начал разрабатываться \textbf{10} лет назад (устаревший). Я обновил версию языка и фреймворка до современных, восстановил работу API сервисов.
\end{itemize}



\cvsection{Проекты}

\cvproject{СтудОко \cvtag{Kotlin}}{Web-приложение для кафедры АТП МФТИ}{Март 2023 -- по н.в.}{}{}

\begin{itemize}
	\item Проект на стадии разработки MVP;
	\item Позволит преподавателям регистрировать курсы и выкладывать домашние задания, а студентам --- подписываться на курсы и получать необходимые доступы на \textbf{GitLab} и \textbf{кластер серверов МФТИ};
	\item Технологии: Kotlin, Ktor, PostgresQL, Docker, Gradle.
\end{itemize}

\divider

\cvproject{SQL to CypherQL Converter \cvtag{C++}}{Open-Source проект -- Инструмент для миграции данных}{Февраль 2022 -- по н.в.}{sql2cypher}{https://github.com/temikfart/sql2cypher/}

\begin{itemize}
	\item Трансляция запросов Microsoft SQL (реляционная СУБД) в язык CypherQL (графовая СУБД neo4j);
	\item Технологии: git-flow, GoogleTest, Doxygen, CI, graphviz, CMake;
	\item Реализовал конфигурацию и логирование приложения, синтаксический анализ запросов, а также переводчик запросов;
	\item Инструмент уже позволяет транслировать несложные схемы, а работа над ним продолжается в дипломной работе.
\end{itemize}

%%%%% Temporarily commented out
\iffalse
\divider

\cvproject{Эмулятор PDP-11 \cvtag{C}}{Курсовая работа в МФТИ}{Апрель -- Июль 2021}{pdp11-emulator}{https://github.com/temikfart/pdp11-emulator/}

\begin{itemize}
	\item Работа была сделана в соотстветсвии с техническим заданием;
	\item Был разработан обширная функциональность: множество команд для исполнения, слова состояния процессора и т.д.
\end{itemize}
\fi
%%%%%


%%%%% Temporarily commented out
\iffalse
\cvsection{Сертификаты и Дипломы}

\cvachievement{\faGraduationCap}{Участие в двух \textbf{международных} олимпиадах по астрономии}{IOAA и IAO}

\cvachievement{\faMedal}{Успешное завершение курса \cvtag{C++}}{Coursera -- <<Основы разработки на C++: Жёлтый пояс>>}

\cvachievement{\faMedal}{Успешное завершение курса \cvtag{Agile}}{Coursera -- <<Agile with Atlassian Jira>>}

\cvachievement{\faMedal}{Успешное завершение курса \cvtag{Java}}{JetBrains Academy -- <<Java for beginners>>}
\fi
%%%%%



%% Switch to the right column. This will now automatically move to the second
%% page if the content is too long.
\switchcolumn

\cvsection{Образование}

\cvevent{\faUniversity Высшее образование}{МФТИ, Бакалавриат}{Сентябрь, 2020 -- Июль, 2024}{}
$4^\text{ый}$ курс физтех-школы прикладной математики и информатики



\cvsection{Навыки}

\begin{itemize}
	\item \textcolor{emphasis}{Языки разработки}
\end{itemize}

\cvtag{Java} -- промышленная разработка
\vspace{1ex}

\cvtag{Kotlin} -- web-приложения на \cvtag{Ktor}
\vspace{1ex}

\cvtag{C++} -- LLVM, STL, Standard library и т.д.
\vspace{1ex}

\cvtag{C} -- Standard, POSIX и MPI libraries.
\vspace{1ex}

\cvtag{Python} -- авто Web UI-тесты с помощью WebDriver (PyTest, unittest)
\vspace{1ex}

\cvtag{Ruby} -- базовые знания языка и создание простых web-приложений на \cvtag{Ruby on Rails}
%\vspace{1ex}

\cvtag{Shell}\cvtag{Go} -- базовые знания
\vspace{.5ex}

\divider

\begin{itemize}
	\item \textcolor{emphasis}{Имел опыт работы}
\end{itemize}

\cvtag{LLVM} -- написание анализирующих pass'ов для \textbf{LLVM Opt}, знание базовых \textbf{LLVM IR} инструкций
\vspace{1ex}

\cvtag{SQL} -- \cvtag{Microsoft SQLS}\cvtag{PostgresQL}
\vspace{-.2ex}

\cvtag{CypherQL} -- \cvtag{Neo4j} (графовая СУБД)
\vspace{1ex}

\cvtag{Gradle} -- сборка проектов с добавлением собственных задач (gradle tasks)
\vspace{1ex}

\cvtag{CMake}\cvtag{Make} -- кроссплатформенная компиляция C, C++ и LaTex проектов
\vspace{1ex}

\cvtag{GoogleTest} -- unit-тестирование (C++)
\vspace{1ex}

\cvtag{Doxygen} -- документирование кода
\vspace{1ex}

\cvtag{PDP-11 Assembly}\cvtag{Bash}\cvtag{MD}\cvtag{LaTex}
\vspace{1ex}

\divider

\begin{itemize}
	\item\textcolor{emphasis}{Рабочие инструменты}
\end{itemize}

\cvtag{GitHub}\cvtag{GitLab}\cvtag{Jira}\cvtag{Agile}\cvtag{Postman}\cvtag{pgAdmin4}



\cvsection{Языки}

\begin{itemize}
	\item Английский --- \cvtag{Pre-Intermediate}
\end{itemize}

\end{paracol}

\end{document}
